\begin{DoxyAuthor}{Autor}
Hansenclever Bassani
\end{DoxyAuthor}
As três classes principais são\-:

\hyperlink{class_parameters}{Parameters}\-: define um conjunto de parâmetros

\hyperlink{class_parameter}{Parameter}\-: define um parâmetro

\hyperlink{class_c_f_g_file}{C\-F\-G\-File}\-: define um arquivo onde os parâmetros serão armazenados.

Exemplo de utilização\-: 
\begin{DoxyCode}
\textcolor{comment}{//Primeiramente é necessário definir uma classe derivada de Parameters que irá conter os parâmetros:}
 
\textcolor{keyword}{class }MyParameters: \textcolor{keyword}{public} \hyperlink{class_parameters}{Parameters} \{

\textcolor{keyword}{public}:
     \textcolor{comment}{//Esta classe contém três parâmetros:}
     \hyperlink{class_parameter}{Parameter<double>} p1; \textcolor{comment}{//parametro double}
     \hyperlink{class_parameter}{Parameter<float>} p2;  \textcolor{comment}{//parametro float}
     \hyperlink{class_parameter}{Parameter<int>} p3;    \textcolor{comment}{//parametro inteiro}

     \textcolor{comment}{//Em seguida definimos um construtor da seguinte maneira:}
     MyParameters() \{
         \textcolor{comment}{//Um nome para este cojunto de parâmetros.}
         section = \textcolor{stringliteral}{"Meus Parametros"};

         \textcolor{comment}{//Uma descrição mais detalhada para este conjunto de parâmetros.}
         comments = \textcolor{stringliteral}{"Parametros do meu sistema p1 p2 e p3"};

         \textcolor{comment}{//Indica quais parâmetros serão salvos em arquivo e um comentário }
         \textcolor{comment}{//explicativo para cada parâmetro que será incluído no arquivo.}
         addParameterD(p1, \textcolor{stringliteral}{"Este parametro controla 1"});
         addParameterD(p2, \textcolor{stringliteral}{"Este parametro controla 2"});
         addParameterD(p3, \textcolor{stringliteral}{"Este parametro controla 3"});

         \textcolor{comment}{//Define valores padrão para os parâmetros}
         p1 = 10.0;
         p2 = 100.;
         p3 = 1000;
     \}
\};

 \textcolor{comment}{//Agora podemos utilizar nossos parâmetros e savá-los em arquivo:}
 \textcolor{keywordtype}{int} main(\textcolor{keywordtype}{void})
 \{
     \textcolor{comment}{//Instancia um arquivo de configuração}
     \hyperlink{class_c_f_g_file}{CFGFile} cfgFile(\textcolor{stringliteral}{"test.cfg"});

     \textcolor{comment}{//Instancia um objeto do tipo MyParameters}
     MyParameters params;

     \textcolor{comment}{//Se o arquivo já existe}
     \textcolor{keywordflow}{if} (cfgFile.exists())
         cfgFile >> params; \textcolor{comment}{//Le os valores atuais}
     \textcolor{keywordflow}{else} 
         cfgFile << params; \textcolor{comment}{//Caso contrário, grava um com os valores padrão}

     \textcolor{comment}{//Imprime o arquivo na tela}
     cout << params;

     \textcolor{comment}{//Imprime apenas alguns parâmetros}
     cout << params.p1.name << \textcolor{stringliteral}{": "} << params.p1 << \textcolor{stringliteral}{"\(\backslash\)t"} << params.p2.name << \textcolor{stringliteral}{": "} << params.p2 << endl;

     \textcolor{comment}{//Le um valor para um parâmetro}
     cout << \textcolor{stringliteral}{"Digite um novo valor para p1:"};
     cin >> params.p1;
     cout << \textcolor{stringliteral}{"Agora o valor de p1 é:"} << params.p1 << endl;

     \textcolor{comment}{//Modifica o valor de outro parâmetro}
     params.p2 = params.p2 + 20;
     cout << \textcolor{stringliteral}{"Agora o valor de p2 é:"} << params.p2 << endl;

     \textcolor{comment}{//Limpa o arquivo e grava os novos parâmetros}
     cfgFile.erase() << params;
\}
\end{DoxyCode}
 